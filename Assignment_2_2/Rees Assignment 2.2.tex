\documentclass[11pt]{article}

% --- Essential Packages ---
\usepackage[utf8]{inputenc}
\usepackage[margin=1in]{geometry}
\usepackage{amsmath, amssymb} % For thermodynamic equations
\usepackage{graphicx}         % For including your Python plots
\usepackage{booktabs}        % For elegant tables
\usepackage{caption}
\usepackage{subcaption}      % To put plots side-by-side
\usepackage{hyperref}
\hypersetup{
    colorlinks=true,
    linkcolor=blue,
    filecolor=magenta,      
    urlcolor=cyan,
}

% --- Document Information ---
\title{ME Grad Assignment 2.2: H2-O2 Fuel Cell Thermodynamics}
\author{Ben Rees}
\date{\today}

\begin{document}

\maketitle

\section{Introduction}
\hspace{5mm}This report analyzes the reversible potential of a hydrogen-oxygen fuel cell using variable specific heat models 
(NIST Shomate Equations)\cite{nist} and a constant $C_p$ model for lower temperature regimes.

\section{Thermodynamic Results}

\subsection{Potential vs. Temperature}
\hspace{5mm}Figure \ref{fig:q1_plot} shows the reversible potential $E_{rev}$ as a function of temperature. We observe a decrease in potential
 as temperature increases due to the negative entropy of the reaction $\Delta S_{rxn}$.

\begin{figure}[htbp]
    \centering
    \includegraphics[width=0.8\textwidth]{Q1_Potential_vs_T.png}
    \caption{Reversible potential from 25$^\circ$C to 800$^\circ$C at $P=1$ atm.}
    \label{fig:q1_plot}
\end{figure}

\subsection{Potential vs. Pressure}
\hspace{5mm}As shown in Figure \ref{fig:q2_plot}, the potential increases logarithmically with pressure, consistent with the Nernst relationship.

\begin{figure}[htbp]
    \centering
    \includegraphics[width=0.8\textwidth]{Q2_Potential_vs_P.png}
    \caption{Reversible potential from 0.1 to 100 atm at $T=650$ K.}
    \label{fig:q2_plot}
\end{figure}

\section{Mathematical Implementation}
\hspace{5mm}The potential was calculated using the Gibbs Free Energy relationship:
\begin{equation}
    E_{rev} = -\frac{\Delta G_{rxn}}{nF} = -\frac{\Delta H - T\Delta S}{nF}.
\end{equation}
\hspace{5mm}The pressure dependence was incorporated using the Nernst equation:
\begin{equation}
    E_{rev}(P) = E_{rev}^0 - \frac{RT}{nF} \ln\left(\frac{P_{H_2} P_{O_2}^{1/2}}{P_{H_2O}}\right).
\end{equation}

\section{Conclusion}
\hspace{5mm}The thermodynamic model successfully captures the voltage drop at higher temperatures and the logarithmic gains with pressure. 

\section{Code Availability}
\hspace{5mm} All Python scripts, thermodynamic coefficients, and diagnostic tools used for this analysis are 
available on GitHub at: \url{https://github.com/Ben-JaminRees/Fuel-Cell-Course-Spring-2026}.

% --- The Bibliography Section ---
\begin{thebibliography}{1}

\bibitem{nist}
P. J. Linstrom and W. G. Mallard, Eds., \textit{NIST Chemistry WebBook, NIST Standard Reference Database Number 69}. Gaithersburg, MD: National Institute of Standards and Technology. [Online]. Available: https://webbook.nist.gov

\end{thebibliography}

\end{document}